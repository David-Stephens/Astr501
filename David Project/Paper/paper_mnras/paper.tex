% mnras_template.tex
%
% LaTeX template for creating an MNRAS paper
%
% v3.0 released 14 May 2015
% (version numbers match those of mnras.cls)
%
% Copyright (C) Royal Astronomical Society 2015
% Authors:
% Keith T. Smith (Royal Astronomical Society)

% Change log
%
% v3.0 May 2015
%    Renamed to match the new package name
%    Version number matches mnras.cls
%    A few minor tweaks to wording
% v1.0 September 2013
%    Beta testing only - never publicly released
%    First version: a simple (ish) template for creating an MNRAS paper

%%%%%%%%%%%%%%%%%%%%%%%%%%%%%%%%%%%%%%%%%%%%%%%%%%
% Basic setup. Most papers should leave these options alone.
% \documentclass[a4paper,fleqn,usenatbib]{mnras}
\documentclass[fleqn,usenatbib]{mnras}

% MNRAS is set in Times font. If you don't have this installed (most LaTeX
% installations will be fine) or prefer the old Computer Modern fonts, comment
% out the following line
%\usepackage{newtxtext,newtxmath}
% Depending on your LaTeX fonts installation, you might get better results with one of these:
%\usepackage{mathptmx}
%\usepackage{txfonts}

% Use vector fonts, so it zooms properly in on-screen viewing software
% Don't change these lines unless you know what you are doing
\usepackage[T1]{fontenc}
\usepackage{ae,aecompl}
\usepackage{subcaption}
\usepackage{graphicx}


%%%%% AUTHORS - PLACE YOUR OWN PACKAGES HERE %%%%%

% Only include extra packages if you really need them. Common packages are:
\usepackage{graphicx}	% Including figure files
\usepackage{amsmath}	% Advanced maths commands
\usepackage{amssymb}	% Extra maths symbols
% macros. please check here before defining something new.
% code.tex
% LaTeX2e macros for naming codes, plus shortcuts for some common ones
% 
\newcommand{\code}[1]{\texttt{#1}}
\newcommand{\flash}{FLASH}
\newcommand{\kepler}{KEPLER}
\newcommand{\nonsmoker}{NON-SMOKER}
\newcommand{\mesa}{\code{MESA}}
\newcommand{\MESA}{\mesa}
\newcommand{\STERN}{STERN}
\newcommand{\ADIPLS}{\code{ADIPLS}}
\newcommand{\DSEP}{DSEP} 

% names for NuGrid codes and code modules
\newcommand{\mppnp}{\code{mppnp}} % multi-zone post-processing network parallel
\newcommand{\ppn}{\code{ppn}}     % refers to the whole NuGrid post-processing code family
\newcommand{\spp}{\code{spp}}     % single-zone ppn


% modules for MESA, from the original instrument paper
\newcommand{\alert}{\code{alert}}
\newcommand{\utils}{\code{utils}}
\newcommand{\const}{\code{const}}
\newcommand{\chem}{\code{chem}}
\newcommand{\diffusion}{\code{diffusion}}
\newcommand{\mtx}{\code{mtx}}
\newcommand{\mesastar}{\mesa~\code{star}}
\newcommand{\MESAstar}{\mesastar}
\newcommand{\num}{\code{num}}
\newcommand{\nuc}{\code{nuc}}
\newcommand{\kap}{\code{kap}}
\newcommand{\eos}{\code{eos}}
\newcommand{\intone}{\code{interp\_1d}}
\newcommand{\inttwo}{\code{interp\_2d}}
\newcommand{\atm}{\code{atm}}
\newcommand{\diff}{\code{diffusion}}
\newcommand{\mlt}{\code{mlt}}
\newcommand{\rates}{\code{rates}}
\newcommand{\net}{\code{net}}
\newcommand{\neu}{\code{neu}}
\newcommand{\weak}{\code{weaklib}}
\newcommand{\screen}{\code{screen}}
\newcommand{\ioniz}{\code{ionization}}
\newcommand{\adipls}{\code{adipls}}
\newcommand{\colors}{\code{colors}}
\newcommand{\reaclib}{\code{reaclib}}
\newcommand{\karo}{\code{karo}}
\newcommand{\astero}{\code{astero}}
\newcommand{\sdk}{\code{SDK}}
\newcommand{\SDK}{\sdk}

% $Id: derivatives.tex 385 2008-07-13 20:07:02Z efb $

%differential operator, roman typeface
\newcommand{\dif}{\ensuremath{\mathrm{d}}}

%derivatives
\newcommand{\D}{{\mathrm d}}
\newcommand{\DD}{{\,\D\!\!\;}}
\newcommand{\ddt}[1]{\frac{\partial #1}{\partial t}} %partial time derivative 
\newcommand{\DDt}[1]{\frac{\dif #1}{\dif t}} %total time derivative
\newcommand{\ddx}[1]{\frac{\partial #1}{\partial x}} %partial derivative wrt x 
\newcommand{\ddy}[1]{\frac{\partial #1}{\partial y}} %partial derivative wrt y 
\newcommand{\DDy}[1]{\frac{\dif #1}{\dif y}} %total derivative wrt y
\newcommand{\ddz}[1]{\frac{\partial #1}{\partial z}} %partial derivative wrt z 
\newcommand{\ppl}[2]{\left(\frac{\partial\ln #1}{\partial\ln
      #2}\right)_{\rho,T}}
\newcommand{\ppll}[2]{\left(\frac{\partial\ln #1}{\partial\ln
      #2}\right)_{\rho,T,\{X_{j\neq i}\}}}
\newcommand{\ddl}[2]{\frac{{\rm d}\ln #1}{{\rm d}\ln #2}}
\newcommand{\DxDy}[2]{{\frac{\D{#1}}{\D{#2}}}}
\newcommand{\dxdy}[2]{{\frac{\partial{#1}}{\partial{#2}}}}
\newcommand{\dxdyind}[3]{{\Brak{\frac{\D{#1}}{\D{#2}}}_{{#3}}}}
\newcommand{\dxdycz}[3]{{\Brak{\frac{\partial{#1}}{\partial{#2}}}_{{#3}}}}
%Misc
\newcommand{\Av}[1]{{\left\langle{#1}\right\rangle}}
\newcommand{\av}[1]{{\langle{#1}\rangle}}
\newcommand{\Frac}[2]{{\Brak{#1}/\Brak{#2}}}
\newcommand{\abs}[1]{{\left|{#1}\right|}}
\input{../paper_resources/macros/nuclides}
\newcommand{\pdcz}{pulse-driven convective zone}

\newcommand{\spr}{\mbox{$s$-process}}
\newcommand{\sprn}{\mbox{$s$ process}}
\newcommand{\ipr}{\mbox{$i$-process}}
\newcommand{\iprn}{\mbox{$i$ process}}
\newcommand{\npr}{\mbox{$n$-process}}
\newcommand{\nprn}{\mbox{$n$ process}}
\newcommand{\rpr}{\mbox{$r$-process}}
\newcommand{\rprn}{\mbox{$r$ process}}
\newcommand{\ppr}{\mbox{$p$-process}}
\newcommand{\pprn}{\mbox{$p$ process}}

% $Id: units.tex 385 2008-07-13 20:07:02Z efb $

%------------------------------------------------------------------------
% typesetting of units
% 
% Note that this doesn't prevent linebreaking between symbol and unit.
% A more sophisticated system is available from CTAN asa units.sty
%------------------------------------------------------------------------

% basic unit typesetteing
\newcommand{\unitspace}{\ensuremath{\,}}
\newcommand{\usp}{\unitspace}
\newcommand{\numberspace}{\ensuremath{\;}}
\newcommand{\nsp}{\numberspace}
\newcommand{\unitstyle}[1]{\ensuremath{\mathrm{#1}}}
\newcommand{\power}[2]{\ensuremath{{#1}^{#2}}}
\newcommand{\natlog}[2]{\ensuremath{#1\times 10^{#2}}} % a*10^b   
\newcommand{\ee}[1]{\ensuremath{\times 10^{#1}}}

% prefixes
\newcommand{\nano}{\unitstyle{n}}
\newcommand{\milli}{\unitstyle{m}}
\newcommand{\centi}{\unitstyle{c}}
\newcommand{\kilo}{\unitstyle{k}}
\newcommand{\Mega}{\unitstyle{M}}
\newcommand{\Giga}{\unitstyle{G}}

% base units, mks
\newcommand{\meter}{\unitstyle{m}}
\newcommand{\kilogram}{\kilo\gram}
\newcommand{\second}{\unitstyle{s}}

\newcommand{\Kelvin}{\unitstyle{K}}
\newcommand{\K}{\Kelvin}  %degrees Kelvin

% base units, cgs
\newcommand{\cm}{\centi\meter}
\newcommand{\gram}{\unitstyle{g}}

% derived units
\newcommand{\grampercc}{\gram\usp\power{\cm}{-3}} %mass density
\newcommand{\grampersquarecm}{\gram\usp\power{\cm}{-2}} %column depth
\newcommand{\squarecmpergram}{\power{\cm}{2}\usp\power{\gram}{-1}} %opacity
\newcommand{\GramPerCc}{\grampercc}
\newcommand{\GramPerSc}{\grampersquarecm}
\newcommand{\columnunit}{\grampersquarecm}
\newcommand{\dyne}{\unitstyle{dyn}} %dyne
\newcommand{\erg}{\unitstyle{ergs}} %ergs
\newcommand{\ergs}{\erg}
\newcommand{\gauss}{\unitstyle{G}} %gauss
\newcommand{\ergspersecond}{\erg\unitspace\power{\second}{-1}}
\newcommand{\ergspergram}{\erg\unitspace\power{\gram}{-1}}
\newcommand{\ergspergs}{\erg\unitspace\power{\gram}{-1}\unitspace\power{\second}{-1}} %angular momentum
\newcommand{\ergssecond}{\erg\unitspace\second}
\newcommand{\cgsflux}{\erg\unitspace\power{\cm}{-2}\usp\power{\second}{-1}}

% Nuclear and atomic units
\newcommand{\amu}{\unitstyle{u}} %atomic mass unit
\newcommand{\angstrom}{\mbox{\AA}} %Angstrom
\newcommand{\fermi}{\unitstyle{fm}} %fermi
\newcommand{\eV}{\unitstyle{eV}}        %eV
\newcommand{\keV}{\kilo\eV} %Kev
\newcommand{\MeV}{\Mega\eV} %MeV

% solar and astronomical units
\newcommand{\Msun}{\ensuremath{\unitstyle{M}_\odot}}
\newcommand{\Lsun}{\ensuremath{\unitstyle{L}_{\odot}}}
\newcommand{\Rsun}{\ensuremath{\unitstyle{R}_{\odot}}}
\newcommand{\Zsun}{\ensuremath{Z_{\odot}}}
\newcommand{\Myr}{\Mega\yr}
\newcommand{\Gyr}{\Giga\yr}
\newcommand{\parsec}{\unitstyle{pc}}
\newcommand{\kpc}{\kilo\parsec} %kiloparsec
\newcommand{\mJy}{\unitstyle{\mu Jy}} %micro Jansky
\newcommand{\Msunyr}{\Msun\,\power{\yr}{-1}}
\newcommand{\MJ}{\ensuremath{\mathrm{M_J}}}
\newcommand{\RJ}{\ensuremath{\mathrm{R_J}}}
\newcommand{\AU}{\unitstyle{AU}}

% misc. units
\newcommand{\minute}{\unitstyle{min}} %minute
\newcommand{\hour}{\unitstyle{hr}} %hour
\newcommand{\yr}{\unitstyle{yr}}        %year
\newcommand{\km}{\kilo\meter}   %kilometers
\newcommand{\Hz}{\unitstyle{Hz}}        %Hertz
\newcommand{\ksec}{\kilo\second} %kilosecond
\newcommand{\kms}{\ensuremath{\mathrm{km}\,\second^{-1}}\xspace}
\newcommand{\vcrit}{{\varv_{\mathrm{crit}}}}
\newcommand{\vkep}{{\varv_{\mathrm{kep}}}}
\newcommand{\vsurf}{{\varv_{\mathrm{surf}}}}
\newcommand{\mol}{\unitstyle{mol}}% mole
\newcommand{\barn}{\unitstyle{b}} %barn

% command to include values
\newcommand{\unit}[2]{\ensuremath{#1\numberspace\mathrm{#2}}}

\input{../paper_resources/macros/vectors}
\input{../paper_resources/macros/formatting}
% symbols for commonly used expressions in the paper.  The point is to define these for 
% consistent notation.  Before defining or using an expression, check that someone hasn't defined it already.
\newcommand{\epsnuc}{\ensuremath{\epsilon_{\mathrm{nuc}}}}	% nuclear heating rate
\newcommand{\epsgrav}{\ensuremath{\epsilon_{\mathrm{grav}}}} % gravitational heating rate
\newcommand{\epsnu}{\ensuremath{\epsilon_{\mathrm{\nu}}}} % neutrino losses
\newcommand{\Teff}{\ensuremath{T_{\!\mathrm{eff}}}}	% effective temperature
\newcommand{\teff}{\Teff}
\newcommand{\Ledd}{\ensuremath{L_{\mathrm{Edd}}}} % Eddington Luminosity
\newcommand{\logg}{\ensuremath{\log g}}	% log surface gravity
\newcommand{\Tc}{\ensuremath{T_{\mathrm{\!c}}}} % central temperature
\newcommand{\Pc}{\ensuremath{P_{\mathrm{\!c}}}} % central pressure
\newcommand{\rhoc}{\ensuremath{\rho_{\mathrm{c}}}} % central density
\newcommand{\CP}{\ensuremath{C_{\!P}}} % specific heat at constant pressure
\newcommand{\Mdot}{\ensuremath{\dot{M}}} % mass-loss rate
\newcommand{\Mc}{\ensuremath{M_{\rm c}}} % core mass
\newcommand{\Mm}{\ensuremath{M_{\rm m}}} % modeled mass
\newcommand{\Rc}{\ensuremath{R_{\rm c}}} % core radius
\newcommand{\Lc}{\ensuremath{L_{\rm c}}} % core luminosity
\newcommand{\Lacc}{\ensuremath{L_{\rm acc}}} % accretion luminosity

% opacity stuff
\newcommand{\kappath}{\ensuremath{\kappa_{\mathrm{th}}}} % opacity for thermal radiation orig.\ in planet
\newcommand{\kappav}{\ensuremath{\kappa_{\mathrm{v}}}} % opacity for irradiation from star

% for correction between baryon densities and mass-energy densities
\newcommand{\nB}{\ensuremath{n_{\mathrm{B}}}}	% baryon density
%
% symbols from the first instrument paper
\newcommand{\alphaMLT}{\ensuremath{\alpha_{\mathrm{MLT}}}}	% mixing length parameter
\newcommand{\chirho}{\ensuremath{\chi_{\rho}}}	% $(\partial\ln P/\partial\ln\rho)_T$
\newcommand{\chiT}{\ensuremath{\chi_{\raisebox{-2pt}{$\scriptstyle T$}}}}	% $(\partial\ln P/\partial\ln T)_{\rho}$
\newcommand{\Gammaone}{\ensuremath{\Gamma_{\!1}}} % $ (\partial\ln P/\partial \ln\rho)_S$
\newcommand{\Dov}{\ensuremath{D_{\mathrm{ov}}}}	% overshoot diffusion coefficient
\newcommand{\nablaad}{\ensuremath{\nabla_{\!\mathrm{ad}}}}	% adiabatic temperature gradient
\newcommand{\nablarad}{\ensuremath{\nabla_{\!\mathrm{rad}}}}	% radiative temperature gradient
\newcommand{\nablaT}{\ensuremath{\nabla_{\!T}}}	% actual temperature gradient
\newcommand{\nablaL}{\ensuremath{\nabla_{\mathrm{\!L}}}}	% Ledoux criterion
\newcommand{\scaleheight}{\ensuremath{\lambda_P}}	% pressure scale height
\newcommand{\Pgas}{\ensuremath{P_{\!\!\mathrm{gas}}}}	% gas pressure
\newcommand{\timestep}{\ensuremath{\delta t}} % numerical timestep
%
% more symbols for radiation and gas pressures
\newcommand{\Prad}{\ensuremath{P_{\!\!\mathrm{rad}}}}	% radiation pressure
\newcommand{\Lrad}{\ensuremath{L_{\mathrm{rad}}}}	% radiative luminosity
\newcommand{\tderiv}[3]{\ensuremath{\left(\frac{\partial #1}{\partial #2}\right)_{#3}}} %thermodynamic derivative
\newcommand{\Lrho}{\ensuremath{L_{\mathrm{inv}}}}	% luminosity at which a density inversion occurs
\newcommand{\Lonset}{\ensuremath{L_{\mathrm{onset}}}}	% luminosity at which the onset of convection occurs
\newcommand{\Fconv}{\ensuremath{F_{\!\mathrm{conv}}}}		% convective flux
\newcommand{\Frad}{\ensuremath{F_{\!\mathrm{rad}}}}	% radiative flux
\newcommand{\supernab}{\ensuremath{\delta_\nabla}}  % superadiabaticity, $\nablaT-\nablaad$
\newcommand{\superthresh}{\ensuremath{\delta_{\nabla,\mathrm{thresh}}}}  % controls when MLT++ is applied
\newcommand{\fsuper}{\ensuremath{f_\nabla}} % reduction factor for $\supernab$
\newcommand{\asuper}{\ensuremath{\alpha_\nabla}}  % smoothing parameter for MLT++
\newcommand{\asupert}{\ensuremath{\widetilde{\asuper}}} % MLT++ parameter used in construction of \asuper
\newcommand{\lambdamax}{\ensuremath{\lambda_{\max}}} % $ \max(\Lrad/\Ledd)$
\newcommand{\betamin}{\ensuremath{\beta_{\min}}} % $ \min(P/\Pgas)$

% mixing symbols
\newcommand{\alphasc}{\ensuremath{\alpha_{\mathrm{sc}}}} % semiconvection efficiency parameter
\newcommand{\alphath}{\ensuremath{\alpha_{\mathrm{th}}}} % thermohaline efficiency parameter
\newcommand{\Dth}{\ensuremath{D_{\mathrm{th}}}} % thermohaline diffusion coefficient

\newcommand{\EFc}{\ensuremath{E_{\mathrm{F,c}}}}  % Fermi energy at center
%
% physical constants
\newcommand{\kB}{\ensuremath{k_\mathrm{B}}} % Boltzmann constant
\newcommand{\NA}{\ensuremath{N_\mathrm{\!A}}} % Avogadro number
\newcommand{\mb}{\ensuremath{m_\mathrm{u}}} % atomic mass unit
\newcommand{\sigmaSB}{\ensuremath{\sigma_\mathrm{\!SB}}} % Stefan-Boltzmann constant

% rotation
\newcommand{\veq}{\ensuremath{\varv_{\mathrm{eq}}}} % equatorial velocity
\newcommand{\veqi}{\ensuremath{\varv_{\mathrm{eq,ini}}}}
\newcommand{\Om}{\ensuremath{\Omega}}  % surface angular velocity
\newcommand{\Omc}{\ensuremath{\Om_{\mathrm{crit}}}} % surface critical angular velocity
\newcommand{\om}{\ensuremath{\omega}}  %  angular velocity
\newcommand{\tkh}{\ensuremath{\tau_{\mathrm{KH}}}} % thermal (Kelvin-Helmholtz) timescale
\newcommand{\LP}{{L_{\mathrm{P}}}} 
\newcommand{\VP}{{V_{\mathrm{P}}}}
\newcommand{\SP}{{S_{\!\mathrm{P}}}} 
\newcommand{\rP}{{r_{\mathrm{P}}}}
\newcommand{\mP}{{m_{\mathrm{P}}}} 
\newcommand{\fP}{{f_{\mathrm{P}}}}
\newcommand{\fT}{{f_{\mathrm{T}}}}

% asteroseismology
\newcommand{\numax}{\ensuremath{\nu_{\mathrm{max}}}} % frequency of maximum power
\newcommand{\dnu}{\ensuremath{\Delta\nu}}  % large frequency separation of pulsation modes
\newcommand{\fov}{\ensuremath{f_{\mathrm{ov}}}} % convective overshoot parameter
\newcommand{\cs}{\ensuremath{c_{\rm s}}} % adiabatic sound speed
\newcommand{\Slamb}{\ensuremath{S_{\!\ell}}} % Lamb frequency


%%%%% AUTHORS - PLACE YOUR OWN COMMANDS HERE %%%%%
% misc. abbreviations
\newcommand{\paperone}{Paper~I} % the first paper

% Please keep new commands to a minimum, and use \newcommand not \def to avoid
% overwriting existing commands. Example:
\newcommand{\pcm}{\,cm$^{-2}$}	% per cm-squared

%%%%%%%%%%%%%%%%%%%%%%%%%%%%%%%%%%%%%%%%%%%%%%%%%%


%%%%%%%%%%%%%%%%%%%%%%%%%%%%%%%%%%%%%%%%%%%%%%%%%%

%% NuGrid/MESA way
%%
% see https://en.wikibooks.org/wiki/LaTeX/Colors for selection of
% colors, such as Apricot Aquamarine BlueGreen BurntOrange
% CornflowerBlue Emerald Gray Lavender Maroon NavyBlue Orchid Plum Red
% RoyalBlue SeaGreen Tan Violet YellowOrange

% for comments
\newcommand{\shortcomment}[3]{\textcolor{#1}{[#2: #3]}}
% for the Ready-To-Read sign off
\newcommand{\rtr}[1]{\shortcomment{red}{RTR}{#1}}
% for freezing a section
\newcommand{\sectionisfrozen}[1]{\textcolor{BlueViolet}{\textsf{\bfseries[section is frozen: send comments to #1]}}}
\newcommand{\sectionisdone}{\textcolor{Green}{\textsf{\bfseries[section is done]}}}

% if you want to embed comments in the text, clone the following with
% a color and your initials...

% for author's comments; choose your own color and replace 'fh' and
% 'FH' with your name and initials
\newcommand{\fhcom}[1]{\shortcomment{PineGreen}{FH}{#1}}

%%%%%%%%%%%%%%%%%%% TITLE PAGE %%%%%%%%%%%%%%%%%%%

% Title of the paper, and the short title which is used in the headers.
% Keep the title short and informative.
\title[Short title, max. 45 characters]{Ok Then, here it is}

% The list of authors, and the short list which is used in the headers.
% If you need two or more lines of authors, add an extra line using \newauthor
\author[D. Stephens]{
David Stephens}


% These dates will be filled out by the publisher
%\date{Accepted XXX. Received YYY; in original form ZZZ}

% Enter the current year, for the copyright statements etc.
%\pubyear{2015}

% Don't change these lines
\begin{document}
\label{firstpage}
\pagerange{\pageref{firstpage}--\pageref{lastpage}}
\maketitle

% Abstract of the paper
\begin{abstract}
During the asymptotic giant branch (AGB) phase of stellar evolution for a 2\Msun~star, the periodic thermal pulses have temperatures as high as 2.9\ee{8} \K. The \neon[22]($\alpha,n$)\magnesium[25]
After the formation of the \carbon[13] pocket 

This is a simple template for authors to write new MNRAS papers.
The abstract should briefly describe the aims, methods, and main results of the paper.
It should be a single paragraph not more than 250 words (200 words for Letters).
No references should appear in the abstract.
\end{abstract}

% Select between one and six entries from the list of approved keywords.
% Don't make up new ones.
\begin{keywords}
AGB -- Diffusion Coefficient -- Pre-solar Grains
\end{keywords}

%%%%%%%%%%%%%%%%%%%%%%%%%%%%%%%%%%%%%%%%%%%%%%%%%%

%%%%%%%%%%%%%%%%% BODY OF PAPER %%%%%%%%%%%%%%%%%%

\section{Introduction}


* Why problem is interesting*

- Zr ratios in pre solar grains from AGB stars show isotopic signatures of material that left the star. 
- This is from envelope and Zr is produced in 13C pocket, then readjusted in the thermal pulse
- Tracer of conditions that were in the star at that time

* what have other people done *

- battino tried varying f and how diffusion happens for effects (couldnt get ratios seen)
- barzyk for xe128 ratio 

* My probelm, the gap! *

- Try a different functionality of diffusion coefficient, hydro leads to this form
- It contains lower diffusion at base of convection zone, less 22Ne reaction, less Zr96

This is a simple template for authors to write new MNRAS papers.
See \texttt{mnras\_sample.tex} for a more complex example, and \texttt{mnras\_guide.tex}
for a full user guide.

All papers should start with an Introduction section, which sets the work
in context, cites relevant earlier studies in the field by \citet{1996ApJ...456..902R},
and describes the problem the authors aim to solve \citep[e.g.][]{cyburt:10}.

\section{Methods and Models}

Normally the next section describes the techniques the authors used.
It is frequently split into subsections, such as Section~\ref{sec:maths} below.

\subsection{\MESA Models}
\label{sec:mesa_models} % used for referring to this section from elsewhere

Only a 2\Msun, $Z=0.02$, stellar evolution model was usedd. This was computed using the \MESA~\citep{mesa} stellar code and taken from the \nugrid NuGrid set1ext, set1.2 models \citep{models}. These models were evolved from the pre-main sequence to a white dwarf. For this work, a singular thermal pulse, the 13th?, was analyzed and a Kippenhahn diagram of this particular thermal pulse can be seen in \Figure{kipp}. 

For these models, the mixing lengths theory \citep{cox}, MLT, is used for the convection zones with a mixing length, $\alphaMLT=1.73$. Overshoot is implemented in \MESA with the formula from \citet{overshoot} and \citet{freytag}:

\begin{equation}
D = D_{0} exp^{-2z/f H_{p0}}
\label{eq:overshoot}
\end{equation} 

Where D$_{0}$ and H$_{p0}$ are taken at the convective boundary. 


\subsection{\zirconium[95] and \iodine[128] Branching}
\label{sec:branching}


\subsection{Diffusion Coefficient Modifications}
\label{sec:diffusion}

\subsection{\mppnp~Post Processing}
\label{sec:mppnp}

\subsection{Neutron Density and Temperature}
\label{sec:neutron}

\section{Results}
Simple mathematics can be inserted into the flow of the text e.g. $2\times3=6$
or $v=220$\,km\,s$^{-1}$, but more complicated expressions should be entered
as a numbered equation:

\begin{equation}
    x=\frac{-b\pm\sqrt{b^2-4ac}}{2a}.
	\label{eq:quadratic}
\end{equation}

Refer back to them as e.g. equation~(\ref{eq:quadratic}).

% Kippenhahn figure
\begin{figure}
  \includegraphics[width=\columnwidth]{figs/2M_Kippenhahn.png}
  \caption{Within the He intershell region, the \carbon[13] pocket forms and isotopic ratios are set by the \spr~(\Fig{zr_ratio13C}). It is then mixed and diluted in the He-flash pulse driven convection zone where temperatures get high enough to activate \neon[22]($\alpha,n$)\magnesium[25]. This shifts the isotopic ratios to those shown in \Figure{zr_ratioDil}} 
  \lFig{kipp}
\end{figure}

% c13 pocket zr ratio
\begin{figure}
  \includegraphics[width=\columnwidth]{figs/C13_Zr.png}
  \caption{Within the \carbon[13] pocket there is a significant amount of \zirconium[94] produced but almost no \zirconium[96]. This is due to the low neutron densities not opening the \zirconium[95] branch.} 
  \lFig{zr_ratio13C}
\end{figure}

% Zr ratio after pulse
\begin{figure}
  \includegraphics[width=\columnwidth]{figs/Pulse_Zr.png}
  \caption{After the \carbon[13] pocket is mixed into the He-flash pulse driven convection zone, the temperatures get high enough for the \neon[22]($\alpha,n$)\magnesium[25] reaction. This provides high enough neutron densities to open the \zirconium[95] branch and boost the \zirconium[96].} 
  \lFig{zr_ratioDil}
\end{figure}

% Xe plot
\begin{figure}
	\includegraphics[width=\columnwidth]{figs/xe.png}
    \caption{Another figure.}
    \lFig{mass-O16}
\end{figure}

% Zr 95 branch
\begin{figure}
	\includegraphics[width=\columnwidth]{figs/Zr95_branch.png}
    \caption{Another figure.}
    \lFig{zr95}
\end{figure}

% Zr 94 plot
\begin{figure}
  \includegraphics[width=\columnwidth]{figs/Zr94.png}
  \caption{This is the Zr94 mil plot \citet{zr}.} 
  \lFig{zr94}
\end{figure}

% diffusion plot
\begin{figure}
  \includegraphics[width=\columnwidth]{figs/Diffusion_compare.png}
  \caption{This is a comparison between the diffusion coefficients} 
  \lFig{diffusion}
\end{figure}

% time scale plot
\begin{figure}
  \includegraphics[width=\columnwidth]{figs/Time_scale.png}
  \caption{This is the time scale plot} 
  \lFig{time_scale}
\end{figure}

% neutron denisty time plot
\begin{figure}
  \includegraphics[width=\columnwidth]{figs/Neutron_Density_Time.png}
  \caption{This is the neutron density as a function of time} 
  \lFig{density}
\end{figure}

% sub time step neutron plots
\begin{figure}
\centering
\begin{subfigure}[b]{1\columnwidth}
   \includegraphics[width=1\columnwidth]{figs/T_Neutron_sub.png}
   \lFig{neutron_sub}
\end{subfigure}

\begin{subfigure}[b]{1\columnwidth}
   \includegraphics[width=1\columnwidth]{figs/T_Neutron_Sub_Closeup}
   \lFig{neutron_sub_close}
\end{subfigure}
\caption{Testing}
\end{figure}

Another results is also good \Fig{mass-O16}.
% Example table
\begin{table}
	\centering
	\caption{This is an example table. Captions appear above each table.
	Remember to define the quantities, symbols and units used.}
	\lTab{example_table}
	\begin{tabular}{lccr} % four columns, alignment for each
		\hline
		A & B & C & D\\
		\hline
		1 & 2 & 3 & 4\\
		2 & 4 & 6 & 8\\
		3 & 5 & 7 & 9\\
		\hline
	\end{tabular}
\end{table}

\section{Conclusions}

The last numbered section should briefly summarise what has been done, and describe
the final conclusions which the authors draw from their work.

\section*{Acknowledgements}

The Acknowledgements section is not numbered. Here you can thank helpful
colleagues, acknowledge funding agencies, telescopes and facilities used etc.
Try to keep it short.

%%%%%%%%%%%%%%%%%%%%%%%%%%%%%%%%%%%%%%%%%%%%%%%%%%

%%%%%%%%%%%%%%%%%%%% REFERENCES %%%%%%%%%%%%%%%%%%

% The best way to enter references is to use BibTeX:

\bibliographystyle{mnras}
\bibliography{paper.bib} % if your bibtex file is
                                             % called example.bib



%%%%%%%%%%%%%%%%%%%%%%%%%%%%%%%%%%%%%%%%%%%%%%%%%%

%%%%%%%%%%%%%%%%% APPENDICES %%%%%%%%%%%%%%%%%%%%%

\appendix

\section{Some extra material}

If you want to present additional material which would interrupt the
flow of the main paper, it can be placed in an Appendix which appears
after the list of references.

%%%%%%%%%%%%%%%%%%%%%%%%%%%%%%%%%%%%%%%%%%%%%%%%%%


% Don't change these lines
\bsp	% typesetting comment
\label{lastpage}
\end{document}

% End of mnras_template.tex
