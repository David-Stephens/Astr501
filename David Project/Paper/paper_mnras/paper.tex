% mnras_template.tex
%
% LaTeX template for creating an MNRAS paper
%
% v3.0 released 14 May 2015
% (version numbers match those of mnras.cls)
%
% Copyright (C) Royal Astronomical Society 2015
% Authors:
% Keith T. Smith (Royal Astronomical Society)

% Change log
%
% v3.0 May 2015
%    Renamed to match the new package name
%    Version number matches mnras.cls
%    A few minor tweaks to wording
% v1.0 September 2013
%    Beta testing only - never publicly released
%    First version: a simple (ish) template for creating an MNRAS paper

% there are 25 thermal pulses, one i am looking at is the 24th

%%%%%%%%%%%%%%%%%%%%%%%%%%%%%%%%%%%%%%%%%%%%%%%%%%
% Basic setup. Most papers should leave these options alone.
% \documentclass[a4paper,fleqn,usenatbib]{mnras}
\documentclass[fleqn,usenatbib]{mnras}

% MNRAS is set in Times font. If you don't have this installed (most LaTeX
% installations will be fine) or prefer the old Computer Modern fonts, comment
% out the following line
%\usepackage{newtxtext,newtxmath}
% Depending on your LaTeX fonts installation, you might get better results with one of these:
%\usepackage{mathptmx}
%\usepackage{txfonts}

% Use vector fonts, so it zooms properly in on-screen viewing software
% Don't change these lines unless you know what you are doing
\usepackage[T1]{fontenc}
\usepackage{ae,aecompl}
\usepackage{subcaption}
\usepackage{graphicx}


%%%%% AUTHORS - PLACE YOUR OWN PACKAGES HERE %%%%%

% Only include extra packages if you really need them. Common packages are:
\usepackage{graphicx}	% Including figure files
\usepackage{amsmath}	% Advanced maths commands
\usepackage{amssymb}	% Extra maths symbols
% macros. please check here before defining something new.
% code.tex
% LaTeX2e macros for naming codes, plus shortcuts for some common ones
% 
\newcommand{\code}[1]{\texttt{#1}}
\newcommand{\flash}{FLASH}
\newcommand{\kepler}{KEPLER}
\newcommand{\nonsmoker}{NON-SMOKER}
\newcommand{\mesa}{\code{MESA}}
\newcommand{\MESA}{\mesa}
\newcommand{\STERN}{STERN}
\newcommand{\ADIPLS}{\code{ADIPLS}}
\newcommand{\DSEP}{DSEP} 

% names for NuGrid codes and code modules
\newcommand{\mppnp}{\code{mppnp}} % multi-zone post-processing network parallel
\newcommand{\ppn}{\code{ppn}}     % refers to the whole NuGrid post-processing code family
\newcommand{\spp}{\code{spp}}     % single-zone ppn


% modules for MESA, from the original instrument paper
\newcommand{\alert}{\code{alert}}
\newcommand{\utils}{\code{utils}}
\newcommand{\const}{\code{const}}
\newcommand{\chem}{\code{chem}}
\newcommand{\diffusion}{\code{diffusion}}
\newcommand{\mtx}{\code{mtx}}
\newcommand{\mesastar}{\mesa~\code{star}}
\newcommand{\MESAstar}{\mesastar}
\newcommand{\num}{\code{num}}
\newcommand{\nuc}{\code{nuc}}
\newcommand{\kap}{\code{kap}}
\newcommand{\eos}{\code{eos}}
\newcommand{\intone}{\code{interp\_1d}}
\newcommand{\inttwo}{\code{interp\_2d}}
\newcommand{\atm}{\code{atm}}
\newcommand{\diff}{\code{diffusion}}
\newcommand{\mlt}{\code{mlt}}
\newcommand{\rates}{\code{rates}}
\newcommand{\net}{\code{net}}
\newcommand{\neu}{\code{neu}}
\newcommand{\weak}{\code{weaklib}}
\newcommand{\screen}{\code{screen}}
\newcommand{\ioniz}{\code{ionization}}
\newcommand{\adipls}{\code{adipls}}
\newcommand{\colors}{\code{colors}}
\newcommand{\reaclib}{\code{reaclib}}
\newcommand{\karo}{\code{karo}}
\newcommand{\astero}{\code{astero}}
\newcommand{\sdk}{\code{SDK}}
\newcommand{\SDK}{\sdk}

% $Id: derivatives.tex 385 2008-07-13 20:07:02Z efb $

%differential operator, roman typeface
\newcommand{\dif}{\ensuremath{\mathrm{d}}}

%derivatives
\newcommand{\D}{{\mathrm d}}
\newcommand{\DD}{{\,\D\!\!\;}}
\newcommand{\ddt}[1]{\frac{\partial #1}{\partial t}} %partial time derivative 
\newcommand{\DDt}[1]{\frac{\dif #1}{\dif t}} %total time derivative
\newcommand{\ddx}[1]{\frac{\partial #1}{\partial x}} %partial derivative wrt x 
\newcommand{\ddy}[1]{\frac{\partial #1}{\partial y}} %partial derivative wrt y 
\newcommand{\DDy}[1]{\frac{\dif #1}{\dif y}} %total derivative wrt y
\newcommand{\ddz}[1]{\frac{\partial #1}{\partial z}} %partial derivative wrt z 
\newcommand{\ppl}[2]{\left(\frac{\partial\ln #1}{\partial\ln
      #2}\right)_{\rho,T}}
\newcommand{\ppll}[2]{\left(\frac{\partial\ln #1}{\partial\ln
      #2}\right)_{\rho,T,\{X_{j\neq i}\}}}
\newcommand{\ddl}[2]{\frac{{\rm d}\ln #1}{{\rm d}\ln #2}}
\newcommand{\DxDy}[2]{{\frac{\D{#1}}{\D{#2}}}}
\newcommand{\dxdy}[2]{{\frac{\partial{#1}}{\partial{#2}}}}
\newcommand{\dxdyind}[3]{{\Brak{\frac{\D{#1}}{\D{#2}}}_{{#3}}}}
\newcommand{\dxdycz}[3]{{\Brak{\frac{\partial{#1}}{\partial{#2}}}_{{#3}}}}
%Misc
\newcommand{\Av}[1]{{\left\langle{#1}\right\rangle}}
\newcommand{\av}[1]{{\langle{#1}\rangle}}
\newcommand{\Frac}[2]{{\Brak{#1}/\Brak{#2}}}
\newcommand{\abs}[1]{{\left|{#1}\right|}}
\input{../paper_resources/macros/nuclides}
\newcommand{\pdcz}{pulse-driven convective zone}

\newcommand{\spr}{\mbox{$s$-process}}
\newcommand{\sprn}{\mbox{$s$ process}}
\newcommand{\ipr}{\mbox{$i$-process}}
\newcommand{\iprn}{\mbox{$i$ process}}
\newcommand{\npr}{\mbox{$n$-process}}
\newcommand{\nprn}{\mbox{$n$ process}}
\newcommand{\rpr}{\mbox{$r$-process}}
\newcommand{\rprn}{\mbox{$r$ process}}
\newcommand{\ppr}{\mbox{$p$-process}}
\newcommand{\pprn}{\mbox{$p$ process}}

% $Id: units.tex 385 2008-07-13 20:07:02Z efb $

%------------------------------------------------------------------------
% typesetting of units
% 
% Note that this doesn't prevent linebreaking between symbol and unit.
% A more sophisticated system is available from CTAN asa units.sty
%------------------------------------------------------------------------

% basic unit typesetteing
\newcommand{\unitspace}{\ensuremath{\,}}
\newcommand{\usp}{\unitspace}
\newcommand{\numberspace}{\ensuremath{\;}}
\newcommand{\nsp}{\numberspace}
\newcommand{\unitstyle}[1]{\ensuremath{\mathrm{#1}}}
\newcommand{\power}[2]{\ensuremath{{#1}^{#2}}}
\newcommand{\natlog}[2]{\ensuremath{#1\times 10^{#2}}} % a*10^b   
\newcommand{\ee}[1]{\ensuremath{\times 10^{#1}}}

% prefixes
\newcommand{\nano}{\unitstyle{n}}
\newcommand{\milli}{\unitstyle{m}}
\newcommand{\centi}{\unitstyle{c}}
\newcommand{\kilo}{\unitstyle{k}}
\newcommand{\Mega}{\unitstyle{M}}
\newcommand{\Giga}{\unitstyle{G}}

% base units, mks
\newcommand{\meter}{\unitstyle{m}}
\newcommand{\kilogram}{\kilo\gram}
\newcommand{\second}{\unitstyle{s}}

\newcommand{\Kelvin}{\unitstyle{K}}
\newcommand{\K}{\Kelvin}  %degrees Kelvin

% base units, cgs
\newcommand{\cm}{\centi\meter}
\newcommand{\gram}{\unitstyle{g}}

% derived units
\newcommand{\grampercc}{\gram\usp\power{\cm}{-3}} %mass density
\newcommand{\grampersquarecm}{\gram\usp\power{\cm}{-2}} %column depth
\newcommand{\squarecmpergram}{\power{\cm}{2}\usp\power{\gram}{-1}} %opacity
\newcommand{\GramPerCc}{\grampercc}
\newcommand{\GramPerSc}{\grampersquarecm}
\newcommand{\columnunit}{\grampersquarecm}
\newcommand{\dyne}{\unitstyle{dyn}} %dyne
\newcommand{\erg}{\unitstyle{ergs}} %ergs
\newcommand{\ergs}{\erg}
\newcommand{\gauss}{\unitstyle{G}} %gauss
\newcommand{\ergspersecond}{\erg\unitspace\power{\second}{-1}}
\newcommand{\ergspergram}{\erg\unitspace\power{\gram}{-1}}
\newcommand{\ergspergs}{\erg\unitspace\power{\gram}{-1}\unitspace\power{\second}{-1}} %angular momentum
\newcommand{\ergssecond}{\erg\unitspace\second}
\newcommand{\cgsflux}{\erg\unitspace\power{\cm}{-2}\usp\power{\second}{-1}}

% Nuclear and atomic units
\newcommand{\amu}{\unitstyle{u}} %atomic mass unit
\newcommand{\angstrom}{\mbox{\AA}} %Angstrom
\newcommand{\fermi}{\unitstyle{fm}} %fermi
\newcommand{\eV}{\unitstyle{eV}}        %eV
\newcommand{\keV}{\kilo\eV} %Kev
\newcommand{\MeV}{\Mega\eV} %MeV

% solar and astronomical units
\newcommand{\Msun}{\ensuremath{\unitstyle{M}_\odot}}
\newcommand{\Lsun}{\ensuremath{\unitstyle{L}_{\odot}}}
\newcommand{\Rsun}{\ensuremath{\unitstyle{R}_{\odot}}}
\newcommand{\Zsun}{\ensuremath{Z_{\odot}}}
\newcommand{\Myr}{\Mega\yr}
\newcommand{\Gyr}{\Giga\yr}
\newcommand{\parsec}{\unitstyle{pc}}
\newcommand{\kpc}{\kilo\parsec} %kiloparsec
\newcommand{\mJy}{\unitstyle{\mu Jy}} %micro Jansky
\newcommand{\Msunyr}{\Msun\,\power{\yr}{-1}}
\newcommand{\MJ}{\ensuremath{\mathrm{M_J}}}
\newcommand{\RJ}{\ensuremath{\mathrm{R_J}}}
\newcommand{\AU}{\unitstyle{AU}}

% misc. units
\newcommand{\minute}{\unitstyle{min}} %minute
\newcommand{\hour}{\unitstyle{hr}} %hour
\newcommand{\yr}{\unitstyle{yr}}        %year
\newcommand{\km}{\kilo\meter}   %kilometers
\newcommand{\Hz}{\unitstyle{Hz}}        %Hertz
\newcommand{\ksec}{\kilo\second} %kilosecond
\newcommand{\kms}{\ensuremath{\mathrm{km}\,\second^{-1}}\xspace}
\newcommand{\vcrit}{{\varv_{\mathrm{crit}}}}
\newcommand{\vkep}{{\varv_{\mathrm{kep}}}}
\newcommand{\vsurf}{{\varv_{\mathrm{surf}}}}
\newcommand{\mol}{\unitstyle{mol}}% mole
\newcommand{\barn}{\unitstyle{b}} %barn

% command to include values
\newcommand{\unit}[2]{\ensuremath{#1\numberspace\mathrm{#2}}}

\input{../paper_resources/macros/vectors}
\input{../paper_resources/macros/formatting}
% symbols for commonly used expressions in the paper.  The point is to define these for 
% consistent notation.  Before defining or using an expression, check that someone hasn't defined it already.
\newcommand{\epsnuc}{\ensuremath{\epsilon_{\mathrm{nuc}}}}	% nuclear heating rate
\newcommand{\epsgrav}{\ensuremath{\epsilon_{\mathrm{grav}}}} % gravitational heating rate
\newcommand{\epsnu}{\ensuremath{\epsilon_{\mathrm{\nu}}}} % neutrino losses
\newcommand{\Teff}{\ensuremath{T_{\!\mathrm{eff}}}}	% effective temperature
\newcommand{\teff}{\Teff}
\newcommand{\Ledd}{\ensuremath{L_{\mathrm{Edd}}}} % Eddington Luminosity
\newcommand{\logg}{\ensuremath{\log g}}	% log surface gravity
\newcommand{\Tc}{\ensuremath{T_{\mathrm{\!c}}}} % central temperature
\newcommand{\Pc}{\ensuremath{P_{\mathrm{\!c}}}} % central pressure
\newcommand{\rhoc}{\ensuremath{\rho_{\mathrm{c}}}} % central density
\newcommand{\CP}{\ensuremath{C_{\!P}}} % specific heat at constant pressure
\newcommand{\Mdot}{\ensuremath{\dot{M}}} % mass-loss rate
\newcommand{\Mc}{\ensuremath{M_{\rm c}}} % core mass
\newcommand{\Mm}{\ensuremath{M_{\rm m}}} % modeled mass
\newcommand{\Rc}{\ensuremath{R_{\rm c}}} % core radius
\newcommand{\Lc}{\ensuremath{L_{\rm c}}} % core luminosity
\newcommand{\Lacc}{\ensuremath{L_{\rm acc}}} % accretion luminosity

% opacity stuff
\newcommand{\kappath}{\ensuremath{\kappa_{\mathrm{th}}}} % opacity for thermal radiation orig.\ in planet
\newcommand{\kappav}{\ensuremath{\kappa_{\mathrm{v}}}} % opacity for irradiation from star

% for correction between baryon densities and mass-energy densities
\newcommand{\nB}{\ensuremath{n_{\mathrm{B}}}}	% baryon density
%
% symbols from the first instrument paper
\newcommand{\alphaMLT}{\ensuremath{\alpha_{\mathrm{MLT}}}}	% mixing length parameter
\newcommand{\chirho}{\ensuremath{\chi_{\rho}}}	% $(\partial\ln P/\partial\ln\rho)_T$
\newcommand{\chiT}{\ensuremath{\chi_{\raisebox{-2pt}{$\scriptstyle T$}}}}	% $(\partial\ln P/\partial\ln T)_{\rho}$
\newcommand{\Gammaone}{\ensuremath{\Gamma_{\!1}}} % $ (\partial\ln P/\partial \ln\rho)_S$
\newcommand{\Dov}{\ensuremath{D_{\mathrm{ov}}}}	% overshoot diffusion coefficient
\newcommand{\nablaad}{\ensuremath{\nabla_{\!\mathrm{ad}}}}	% adiabatic temperature gradient
\newcommand{\nablarad}{\ensuremath{\nabla_{\!\mathrm{rad}}}}	% radiative temperature gradient
\newcommand{\nablaT}{\ensuremath{\nabla_{\!T}}}	% actual temperature gradient
\newcommand{\nablaL}{\ensuremath{\nabla_{\mathrm{\!L}}}}	% Ledoux criterion
\newcommand{\scaleheight}{\ensuremath{\lambda_P}}	% pressure scale height
\newcommand{\Pgas}{\ensuremath{P_{\!\!\mathrm{gas}}}}	% gas pressure
\newcommand{\timestep}{\ensuremath{\delta t}} % numerical timestep
%
% more symbols for radiation and gas pressures
\newcommand{\Prad}{\ensuremath{P_{\!\!\mathrm{rad}}}}	% radiation pressure
\newcommand{\Lrad}{\ensuremath{L_{\mathrm{rad}}}}	% radiative luminosity
\newcommand{\tderiv}[3]{\ensuremath{\left(\frac{\partial #1}{\partial #2}\right)_{#3}}} %thermodynamic derivative
\newcommand{\Lrho}{\ensuremath{L_{\mathrm{inv}}}}	% luminosity at which a density inversion occurs
\newcommand{\Lonset}{\ensuremath{L_{\mathrm{onset}}}}	% luminosity at which the onset of convection occurs
\newcommand{\Fconv}{\ensuremath{F_{\!\mathrm{conv}}}}		% convective flux
\newcommand{\Frad}{\ensuremath{F_{\!\mathrm{rad}}}}	% radiative flux
\newcommand{\supernab}{\ensuremath{\delta_\nabla}}  % superadiabaticity, $\nablaT-\nablaad$
\newcommand{\superthresh}{\ensuremath{\delta_{\nabla,\mathrm{thresh}}}}  % controls when MLT++ is applied
\newcommand{\fsuper}{\ensuremath{f_\nabla}} % reduction factor for $\supernab$
\newcommand{\asuper}{\ensuremath{\alpha_\nabla}}  % smoothing parameter for MLT++
\newcommand{\asupert}{\ensuremath{\widetilde{\asuper}}} % MLT++ parameter used in construction of \asuper
\newcommand{\lambdamax}{\ensuremath{\lambda_{\max}}} % $ \max(\Lrad/\Ledd)$
\newcommand{\betamin}{\ensuremath{\beta_{\min}}} % $ \min(P/\Pgas)$

% mixing symbols
\newcommand{\alphasc}{\ensuremath{\alpha_{\mathrm{sc}}}} % semiconvection efficiency parameter
\newcommand{\alphath}{\ensuremath{\alpha_{\mathrm{th}}}} % thermohaline efficiency parameter
\newcommand{\Dth}{\ensuremath{D_{\mathrm{th}}}} % thermohaline diffusion coefficient

\newcommand{\EFc}{\ensuremath{E_{\mathrm{F,c}}}}  % Fermi energy at center
%
% physical constants
\newcommand{\kB}{\ensuremath{k_\mathrm{B}}} % Boltzmann constant
\newcommand{\NA}{\ensuremath{N_\mathrm{\!A}}} % Avogadro number
\newcommand{\mb}{\ensuremath{m_\mathrm{u}}} % atomic mass unit
\newcommand{\sigmaSB}{\ensuremath{\sigma_\mathrm{\!SB}}} % Stefan-Boltzmann constant

% rotation
\newcommand{\veq}{\ensuremath{\varv_{\mathrm{eq}}}} % equatorial velocity
\newcommand{\veqi}{\ensuremath{\varv_{\mathrm{eq,ini}}}}
\newcommand{\Om}{\ensuremath{\Omega}}  % surface angular velocity
\newcommand{\Omc}{\ensuremath{\Om_{\mathrm{crit}}}} % surface critical angular velocity
\newcommand{\om}{\ensuremath{\omega}}  %  angular velocity
\newcommand{\tkh}{\ensuremath{\tau_{\mathrm{KH}}}} % thermal (Kelvin-Helmholtz) timescale
\newcommand{\LP}{{L_{\mathrm{P}}}} 
\newcommand{\VP}{{V_{\mathrm{P}}}}
\newcommand{\SP}{{S_{\!\mathrm{P}}}} 
\newcommand{\rP}{{r_{\mathrm{P}}}}
\newcommand{\mP}{{m_{\mathrm{P}}}} 
\newcommand{\fP}{{f_{\mathrm{P}}}}
\newcommand{\fT}{{f_{\mathrm{T}}}}

% asteroseismology
\newcommand{\numax}{\ensuremath{\nu_{\mathrm{max}}}} % frequency of maximum power
\newcommand{\dnu}{\ensuremath{\Delta\nu}}  % large frequency separation of pulsation modes
\newcommand{\fov}{\ensuremath{f_{\mathrm{ov}}}} % convective overshoot parameter
\newcommand{\cs}{\ensuremath{c_{\rm s}}} % adiabatic sound speed
\newcommand{\Slamb}{\ensuremath{S_{\!\ell}}} % Lamb frequency


%%%%% AUTHORS - PLACE YOUR OWN COMMANDS HERE %%%%%
% misc. abbreviations
\newcommand{\paperone}{Paper~I} % the first paper

% Please keep new commands to a minimum, and use \newcommand not \def to avoid
% overwriting existing commands. Example:
\newcommand{\pcm}{\,cm$^{-2}$}	% per cm-squared

%%%%%%%%%%%%%%%%%%%%%%%%%%%%%%%%%%%%%%%%%%%%%%%%%%


%%%%%%%%%%%%%%%%%%%%%%%%%%%%%%%%%%%%%%%%%%%%%%%%%%

%% NuGrid/MESA way
%%
% see https://en.wikibooks.org/wiki/LaTeX/Colors for selection of
% colors, such as Apricot Aquamarine BlueGreen BurntOrange
% CornflowerBlue Emerald Gray Lavender Maroon NavyBlue Orchid Plum Red
% RoyalBlue SeaGreen Tan Violet YellowOrange

% for comments
\newcommand{\shortcomment}[3]{\textcolor{#1}{[#2: #3]}}
% for the Ready-To-Read sign off
\newcommand{\rtr}[1]{\shortcomment{red}{RTR}{#1}}
% for freezing a section
\newcommand{\sectionisfrozen}[1]{\textcolor{BlueViolet}{\textsf{\bfseries[section is frozen: send comments to #1]}}}
\newcommand{\sectionisdone}{\textcolor{Green}{\textsf{\bfseries[section is done]}}}

% if you want to embed comments in the text, clone the following with
% a color and your initials...

% for author's comments; choose your own color and replace 'fh' and
% 'FH' with your name and initials
\newcommand{\fhcom}[1]{\shortcomment{PineGreen}{FH}{#1}}

%%%%%%%%%%%%%%%%%%% TITLE PAGE %%%%%%%%%%%%%%%%%%%

% Title of the paper, and the short title which is used in the headers.
% Keep the title short and informative.
\title[Short title, max. 45 characters]{Ok Then, here it is}

% The list of authors, and the short list which is used in the headers.
% If you need two or more lines of authors, add an extra line using \newauthor
\author[D. Stephens]{
David Stephens}


% These dates will be filled out by the publisher
%\date{Accepted XXX. Received YYY; in original form ZZZ}

% Enter the current year, for the copyright statements etc.
%\pubyear{2015}

% Don't change these lines
\begin{document}
\label{firstpage}
\pagerange{\pageref{firstpage}--\pageref{lastpage}}
\maketitle

% Abstract of the paper
\begin{abstract}
During the asymptotic giant branch (AGB) phase of stellar evolution for a 2\Msun~star, the periodic thermal pulses have temperatures as high as 2.9\ee{8} \K. The \neon[22]($\alpha,n$)\magnesium[25]
After the formation of the \carbon[13] pocket 

This is a simple template for authors to write new MNRAS papers.
The abstract should briefly describe the aims, methods, and main results of the paper.
It should be a single paragraph not more than 250 words (200 words for Letters).
No references should appear in the abstract.
\end{abstract}

% Select between one and six entries from the list of approved keywords.
% Don't make up new ones.
\begin{keywords}
AGB -- Diffusion Coefficient -- Pre-solar Grains
\end{keywords}

%%%%%%%%%%%%%%%%%%%%%%%%%%%%%%%%%%%%%%%%%%%%%%%%%%

%%%%%%%%%%%%%%%%% BODY OF PAPER %%%%%%%%%%%%%%%%%%

\section{Introduction}

The Zr isotopes are produced from the \spr with \iron[56] as the seed. To start the \spr~the neutron densities must reach above 1\ee{8}  \power{\centi\meter}{-3} and get past the first peak for many of the Zr isotopes. One of the \spr sites in middle mass stars is during the thermal pulses caused by successive He-flashes during their asymptotic giant branch (AGB) phase. In between these thermal pulses the \carbon[13] pocket develops and through the \carbon[13]($\alpha,n$)\oxygen[16] reaction the \spr can occur. Then, this material will be mixed through a He-flash pulse driven convection zone (PDCZ) and the \neon[22]($\alpha,n$)\magnesium[25] reaction will activate the \spr for a short period of time. The neutron densities that are achieved through these two processes differs significantly and depends on the conditions in the star. The Zr isotopic ratios can be used to understand the conditions within the AGB stars as they are sensitive to the neutron densities that they are exposed to \citep{zr}.

The \zirconium[95] isotope is unstable with a lifetime of 64 days. Within the \carbon[13] pocket, the neutron densities do not get large enough to allow for this branch to open so most of the \zirconium[95] will decay to \molybdenum[95] rather than undergo \zirconium[95]($n,\gamma$)\zirconium[96]. This skews the isotopic ratios of \zirconium[96] / \zirconium[94] as there will be production of \zirconium[94] but almost none of \zirconium[96] during the \carbon[13] pocket formation. However, once the He-flash PDCZ forms, the \neon[22]($\alpha,n$)\magnesium[25] reaction can generate neutron densities as large as 1\ee{10} \power{\centi\meter}{-3} which will open the branch for the \zirconium[95]($n,\gamma$)\zirconium[96] to occur shifting the isotopic ratios again. Many of these thermal pulses happen throughout the AGB phase and some of the material left behind from the He-flash PDCZ can be mixed into the H envelope from the third dredge up that happens after. Due to the significant mass loss during the AGB phase, the isotopic ratios of Zr can be measured directly from the pre-solar SiC grains that form from the material lost during the AGB phase \citep{grain}. \citet{zr} compared the H envelope \zirconium[96] / \zirconium[94] ratios produced in 3\Msun~and 2\Msun~stellar models with the ratios measured in SiC grains from \citet{grain}. The stellar model results overestimated the ratio in all models (\Fig{zr94}).

The goal of this work is to test if minor changes, suggested by hydrodynamic simulations, to the diffusion coefficient as a function of mass will affect the Zr isotopic ratios that the He-flash PDCZ produces. The ratios that are computed with the changes to the diffusion coefficient will be compared to those without and their impact on results that \citep{zr} predicts will be discussed. The analysis will be conducted on a 2\Msun, $Z=0.02$ stellar model.

* Why problem is interesting*

- Zr ratios in pre solar grains from AGB stars show isotopic signatures of material that left the star. 
- This is from envelope and Zr is produced in 13C pocket, then readjusted in the thermal pulse
- Tracer of conditions that were in the star at that time

* what have other people done *

- battino tried varying f and how diffusion happens for effects (couldnt get ratios seen)
- barzyk for xe128 ratio 

* My probelm, the gap! *

- Try a different functionality of diffusion coefficient, hydro leads to this form
- It contains lower diffusion at base of convection zone, less 22Ne reaction, less Zr96

This is a simple template for authors to write new MNRAS papers.
See \texttt{mnras\_sample.tex} for a more complex example, and \texttt{mnras\_guide.tex}
for a full user guide.

All papers should start with an Introduction section, which sets the work
in context, cites relevant earlier studies in the field by \citet{1996ApJ...456..902R},
and describes the problem the authors aim to solve \citep[e.g.][]{cyburt:10}.


\section{Methods and Models}

\subsection{\MESA~Models}
\label{sec:mesa_models} % used for referring to this section from elsewhere

Only a 2\Msun, $Z=0.02$, stellar evolution model was used. This was computed using the \MESA~\citep{mesa} stellar code and taken from the \nugrid NuGrid set1ext, set1.2 models \citep{models}. These models were evolved from the pre-main sequence to a white dwarf. For this work, a singular thermal pulse, the 24$^{th}$, was analyzed and a Kippenhahn diagram of this particular thermal pulse can be seen in \Figure{kipp}. 

For these models, the mixing lengths theory \citep{cox}, MLT, is used for the convection zones with a mixing length, $\alphaMLT=1.73$. Overshoot is implemented in \MESA with the formula from \citet{overshoot} and \citet{freytag}:

\begin{equation}
D = D_{0} exp^{-2z/f H_{p0}}
\label{eq:overshoot}
\end{equation} 

\noindent Where D$_{0}$ and H$_{p0}$ are taken at the convective boundary (\Sect{diffusion}). The models were also post processed with \mppnp as part of the work done in \citep{models}. The methods used in this work to post process the He-flash PDCZ are outlined in \Sect{mppnp}.

\subsection{Neutron Density and Temperature}
\label{sec:neutron}

Within the PDCZ, there is a significant amount in mass fraction of \neon[22] that has been produced as well as a high mass fraction of alpha particles. There are a lot of alpha particles from the H-shell burning products as well as the helium burning is not significant at this point (\Fig{kipp}). Once temperatures reach close to T$_{8} \approx 2.8$ the \neon[22]($\alpha,n$)\magnesium[25] reaction is activated. This can be seen from the high neutron densities shown in \FigTwo{density}{neutron_sub}. When looking at a particular model number, the neutron density as a function of mass within the convective boundaries has a significant peak near the lower boundary. This is due to the \neon[22]($\alpha,n$)\magnesium[25] reaction being very sensitive to temperature and a huge portion of the reactions will take place at the bottom of the convection zone. Then, the neutrons will diffuse and mix throughout the convection zone allowing for the \spr to take place again. 

In \Fig{density} the average neutron density steadily rises until the maximum temperature that will occur within the convection zone and then slowly drops. This peak in the average neutron densities is very short lived as it is only sustained for a few years. However, this will have an impact on the isotopic ratios of \zirconium as discussed in \Sect{branching}. The importance of this is that some of the material from the PDCZ will be mixed to the H envelope from the periodic third dredge-up that occurs between thermal pulses (\Fig{kipp}).

\subsection{Diffusion Coefficient Modifications}
\label{sec:diffusion}

The \MESA models use the MLT and the diffusion equation to estimate how the mass fraction of isotopic species are transported and mixed throughout convection zones. To calculate the change in mass fraction of a particular species at a particular mass coordinate in time, the stellar evolution codes solve the differential equation 
\begin{equation}
\frac{d\mathbf{X_{i}}}{dt} = \left. \frac{d\mathbf{X_{i}}}{dt} \right \rvert_{burn} + \left. \frac{d\mathbf{X_{i}}}{dt} \right \rvert_{mix}
\end{equation}

\noindent which contains the changes due to the possible reactions that the particular species can be involved in and the spatial diffusion of matter due to the mixing in a convection zone. Using the diffusion formalism, this mixing term is given by
\begin{equation}
\left. \frac{d\mathbf{X_{i}}}{dt} \right \rvert_{mix} = \frac{\partial}{\partial m} [(4\pi r^{2} \rho)^{2} D(m) \frac{d\mathbf{X_{i}}}{dm}]
\label{eq:diffusion}
\end{equation} 

\noindent Where $D(m)$ is the diffusion coefficient at mass coordinate m. 

Using the MLT, the diffusion coefficient is given by
\begin{equation}
D = \frac{1}{3} v_{MLT} \alphaMLT H_{p}
\label{eq:diffusion_mlt}
\end{equation}

\noindent Where \alphaMLT$H_{p}$ is called the mixing length. If just using the MLT, this diffusion would immediately go to zero at the Schwarzschild boundary (convective boundary). The Schwarzschild boundary is the mass coordinate where the Schwarzschild criterion, the condition in which the gas will be convectively unstable, is located. The Schwarzschild criterion is when the condition
\begin{equation}
\nabla_{rad} > \nabla_{ad}
\end{equation}

\noindent is satisfied. $\nabla_{rad} = \left(\dxdy{lnT}{lnP}\right)_{rad}$ is the gradient that the star would have if all of the energy is transported by radiation while $\nabla_{ad} = \left(\dxdy{lnT}{lnP}\right)_{ad}$ is the gradient from all of the energy being transported by convection. 

The functionality of the MLT diffusion coefficient across the convection zone is similar to what is calculated from 3D hydrodynamic simulations. However, a main qualitative difference is that the diffusion coefficient begins to drop well before the convective boundary while the MLT diffusion coefficient will not. A model for the diffusion coefficient that has been able to reproduce results from 3D hydro simulations is provided by \citep{4pi}. The formula for the diffusion coefficient is  
\begin{equation}
D_{hydro} = v_{MLT}min(\alphaMLT H_{p}, |r - r_{0}|)
\label{eq:Jones}
\end{equation} 

\noindent where $r_{0} = r_{SC} - f_{CBM}H_{p}^{SC}$ which are quantities that are evaluated at the Schwarzschild boundary. This does not make major changes to the functionality and magnitude of the diffusion coefficient except that it falls off quicker when approaching the convective boundaries. This can be seen in \Figure{diffusion}. With the temperature being highest at the lower convective boundary (\Fig{neutron_sub}) and with using \EQ{Jones} for the diffusion coefficient rather than \EQ{diffusion_mlt}, there would be less matter diffusing to the higher temperatures of the lower boundary. The possible nucleosynthetic consequences of this are discussed in \Sect{branching} and shown in \Sect{zr_zr}.

\subsection{\zirconium[95] and \iodine[128] Branching}
\label{sec:branching}

Within the \carbon[13] pocket, the reaction that causes the high neutron densities for the \spr is \carbon[13]($\alpha,n$)\oxygen[16]. The neutron densities do not reach N$_{n} \approx 5\ee{8}\power{\centi\meter}{-3}$ that are required to open the \zirconium[95] branch \citep{zr}. This causes a situation in which there is production of \zirconium[94] from the \spr but due to low neutron densities not as much \zirconium[96] is produced. This can be seen from the isotope mass fractions of Zirconium in \Fig{zr_ratio13C}. The \spr path and \zirconium[95] branch can be seen in \Figure{zr95}

During the He-flash PDCZ temperatures get high enough for the \neon[22]($\alpha,n$)\magnesium[25] reaction to occur. Neutron densities can exceed the lower limit for the \zirconium[95] branching (\Fig{density}). This will shift the isotopic mass fractions of \zirconium[96] and \zirconium[94] to what is shown in \Figure{zr_ratioDil}. Their mass fractions dropped significantly due to the dilution of the \carbon[13] pocket into the PDCZ however the change in the ratio, log$_{10}$(\zirconium[96] / \zirconium[94]), went from -2.7 to -0.9. This is directly from the opening of the \zirconium[95] branch. This branch is only open for a very short amount of time as the neutron density is only larger than  $5\ee{8}\power{\centi\meter}{-3}$ for a few years (\Fig{density}) but its effects on the Zirconium isotopic ratios are significant. 

To compare with the measurements made by \citep{grain}, the isotopic ratios need to be represented into the per mil, $\delta$, form. For a given isotopic ration, this is defined as
\begin{equation}
\delta\left(\zirconium[96] / \zirconium[94]\right) = [\left(\frac{\zirconium[96]}{\zirconium[94]} / \frac{\zirconium[96]_{\odot}}{\zirconium[94]_{\odot}}\right) - 1]\times 1000
\label{eq:mil}
\end{equation}

\noindent The comparison of $\delta\left(\zirconium[96] / \zirconium[94]\right)$ from the pre-solar grains measured by \citep{grain} and what stellar evolution models predict from \citep{zr} is in \Figure{zr94}.

The \iodine[128] isotope is unstable with a lifetime of only 25 minutes and it has two competing branches. These reactions are the $\beta^{-}$ decay, \iodine[128]($\beta^{-}$)\xenon[128] and the electron capture \iodine[128]($\beta^{+}$)\tellurium[128]. The electron capture branching is very sensitive to the temperature as well as the electron density \citep{reif}. \Figure{xe} shows all of the isotopes that play a role in the \iodine[128] branching. Within the He-flash PDCZ the temperature peaks very close to the lower convective boundary. With the convective time scale (\Sect{mppnp}) during the PDCZ being only a few hours, fresh \iodine[126] from the top of the convection zone will come to the bottom and can capture a neutron easily due to the very high neutron densities there. With the short half-life of \iodine[128],a significant portion of these will decay while being at the higher temperatures near the lower convective boundary. This will lead to a significant portion of the \iondine[128] to undergo \iodine[128]($\beta^{+}$)\tellurium[128] rather than \iodine[128]($\beta^{-}$)\xenon[128]. Since the \ppr is not occurring within the He-flash PDCZ, the only production of \xenon[128] comes from the \iodine[128]($\beta^{-}$)\xenon[128] and the losses are due to neutron captures. Because of this, the \xenon[128] / \xenon[130] ratio can be used as a tracer of what the dominating branch of \iodine[128] is.

\subsection{\mppnp~Post Processing}
\label{sec:mppnp}

The nucleosynthesis that was calculated with the \mppnp code used a network of over 1000 isotopes and was limited by species that had a $\beta$ decay less than 
To compute the nucleosynthesis that is occurring in the PDCZ, the \mppnp code was used. From \citep{models} the nucleosynthesis was already calculated for the 2\Msun, $Z=0.02$, stellar model with the default MLT diffusion coefficient. To calculate models with the diffusion coefficient from \EQ{Jones} the H5 files that the \MESA data was outputted to was modified. This was simply done with python and the \mppnp code could be run normally but it would have the modified diffusion coefficient.  
\section{Results}
\subsection{Effects on $\delta$(\zirconium[96] / \zirconium[94])}
\label{sec:zr_zr}
Simple mathematics can be inserted into the flow of the text e.g. $2\times3=6$
or $v=220$\,km\,s$^{-1}$, but more complicated expressions should be entered
as a numbered equation:

\begin{equation}
    x=\frac{-b\pm\sqrt{b^2-4ac}}{2a}.
	\label{eq:quadratic}
\end{equation}

Refer back to them as e.g. equation~(\ref{eq:quadratic}).

% Kippenhahn figure
\begin{figure}
  \includegraphics[width=\columnwidth]{figs/2M_Kippenhahn.png}
  \caption{Within the He intershell region, the \carbon[13] pocket forms and isotopic ratios are set by the \spr~(\Fig{zr_ratio13C}). It is then mixed and diluted in the He-flash pulse driven convection zone where temperatures get high enough to activate \neon[22]($\alpha,n$)\magnesium[25]. This shifts the isotopic ratios to those shown in \Figure{zr_ratioDil}} 
  \lFig{kipp}
\end{figure}

% c13 pocket zr ratio
\begin{figure}
  \includegraphics[width=\columnwidth]{figs/C13_Zr.png}
  \caption{These isotope ratios are averaged across the \carbon[13] pocket. Within the \carbon[13] pocket there is a significant amount of \zirconium[94] produced but almost no \zirconium[96]. This is due to the low neutron densities not opening the \zirconium[95] branch. The logarithm base 10 of the \zirconium[96] / \zirconium[94] ratio is approximately -2.7} 
  \lFig{zr_ratio13C}
\end{figure}

% Zr ratio after pulse
\begin{figure}
  \includegraphics[width=\columnwidth]{figs/Pulse_Zr.png}
  \caption{These isotope ratios are averaged across the He-flash PDCZ. After the \carbon[13] pocket is mixed into the He-flash pulse driven convection zone, the temperatures get high enough for the \neon[22]($\alpha,n$)\magnesium[25] reaction. This provides high enough neutron densities to open the \zirconium[95] branch and boost the \zirconium[96].The logarithm base 10 of the \zirconium[96] / \zirconium[94] ratio is approximately -0.9} 
  \lFig{zr_ratioDil}
\end{figure}

% Zr 95 branch
\begin{figure}
	\includegraphics[width=\columnwidth]{figs/Zr95_branch.png}
    \caption{In order for there to be significant production of \zirconium[96] the neutron densities must be high enough such that the \zirconium[95] (half-life 64 days) formed from the reaction \zirconium[94]($n,\gamma$)\zirconium[95] does not all decay to \niobium[95].}
    \lFig{zr95}
\end{figure}


% Xe plot
\begin{figure}
	\includegraphics[width=\columnwidth]{figs/xe.png}
    \caption{The unstable isotope, \iodine[128], can either $\beta^{-}$ decay or capture an electron ($\beta^{+}$). The electron capture branching is very sensitive to the temperature. If it $\beta^{-}$ decays there will be production of \xenon[128] and with high neutron densities, \xenon[130] can be produced as well. If it captures an electron, it can capture neutrons and decay to end up at \xenon[130]. This figure is from \citep{reif}.}
    \lFig{xe}
\end{figure}

% Zr 94 plot
\begin{figure}
  \includegraphics[width=\columnwidth]{figs/Zr94.png}
  \caption{This is the Zr94 mil plot \citet{zr}.} 
  \lFig{zr94}
\end{figure}

% diffusion plot
\begin{figure}
  \includegraphics[width=\columnwidth]{figs/Diffusion_compare.png}
  \caption{The MLT diffusion coefficient has a more flat profile between the convective boundaries while the diffusion coefficient given by \citep{4pi} (\Eq{Jones}) is about an order of magnitude smaller at the convective boundaries.} 
  \lFig{diffusion}
\end{figure}

% time scale plot
\begin{figure}
  \includegraphics[width=\columnwidth]{figs/Time_scale.png}
  \caption{This is the time scale plot} 
  \lFig{time_scale}
\end{figure}

% neutron denisty time plot
\begin{figure}
  \includegraphics[width=\columnwidth]{figs/Neutron_Density_Time.png}
  \caption{This is the neutron density as a function of time} 
  \lFig{density}
\end{figure}

% sub time step neutron plots
\begin{figure}
   \includegraphics[width=1\columnwidth]{figs/T_Neutron_sub.png}
   \lFig{neutron_sub}
\end{figure}


\section{Conclusions}

The last numbered section should briefly summarise what has been done, and describe
the final conclusions which the authors draw from their work.

\section*{Acknowledgements}

The Acknowledgements section is not numbered. Here you can thank helpful
colleagues, acknowledge funding agencies, telescopes and facilities used etc.
Try to keep it short.

%%%%%%%%%%%%%%%%%%%%%%%%%%%%%%%%%%%%%%%%%%%%%%%%%%

%%%%%%%%%%%%%%%%%%%% REFERENCES %%%%%%%%%%%%%%%%%%

% The best way to enter references is to use BibTeX:

\bibliographystyle{mnras}
\bibliography{paper.bib} % if your bibtex file is
                                             % called example.bib



%%%%%%%%%%%%%%%%%%%%%%%%%%%%%%%%%%%%%%%%%%%%%%%%%%

%%%%%%%%%%%%%%%%% APPENDICES %%%%%%%%%%%%%%%%%%%%%

\appendix

\section{Some extra material}

If you want to present additional material which would interrupt the
flow of the main paper, it can be placed in an Appendix which appears
after the list of references.

%%%%%%%%%%%%%%%%%%%%%%%%%%%%%%%%%%%%%%%%%%%%%%%%%%


% Don't change these lines
\bsp	% typesetting comment
\label{lastpage}
\end{document}

% End of mnras_template.tex
