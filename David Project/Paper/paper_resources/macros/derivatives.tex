% $Id: derivatives.tex 385 2008-07-13 20:07:02Z efb $

%differential operator, roman typeface
\newcommand{\dif}{\ensuremath{\mathrm{d}}}

%derivatives
\newcommand{\D}{{\mathrm d}}
\newcommand{\DD}{{\,\D\!\!\;}}
\newcommand{\ddt}[1]{\frac{\partial #1}{\partial t}} %partial time derivative 
\newcommand{\DDt}[1]{\frac{\dif #1}{\dif t}} %total time derivative
\newcommand{\ddx}[1]{\frac{\partial #1}{\partial x}} %partial derivative wrt x 
\newcommand{\ddy}[1]{\frac{\partial #1}{\partial y}} %partial derivative wrt y 
\newcommand{\DDy}[1]{\frac{\dif #1}{\dif y}} %total derivative wrt y
\newcommand{\ddz}[1]{\frac{\partial #1}{\partial z}} %partial derivative wrt z 
\newcommand{\ppl}[2]{\left(\frac{\partial\ln #1}{\partial\ln
      #2}\right)_{\rho,T}}
\newcommand{\ppll}[2]{\left(\frac{\partial\ln #1}{\partial\ln
      #2}\right)_{\rho,T,\{X_{j\neq i}\}}}
\newcommand{\ddl}[2]{\frac{{\rm d}\ln #1}{{\rm d}\ln #2}}
\newcommand{\DxDy}[2]{{\frac{\D{#1}}{\D{#2}}}}
\newcommand{\dxdy}[2]{{\frac{\partial{#1}}{\partial{#2}}}}
\newcommand{\dxdyind}[3]{{\Brak{\frac{\D{#1}}{\D{#2}}}_{{#3}}}}
\newcommand{\dxdycz}[3]{{\Brak{\frac{\partial{#1}}{\partial{#2}}}_{{#3}}}}
%Misc
\newcommand{\Av}[1]{{\left\langle{#1}\right\rangle}}
\newcommand{\av}[1]{{\langle{#1}\rangle}}
\newcommand{\Frac}[2]{{\Brak{#1}/\Brak{#2}}}
\newcommand{\abs}[1]{{\left|{#1}\right|}}